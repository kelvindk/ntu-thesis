\chapter{Design Exploration}

The first step of iterative design for a wearable sensing platform is to understand the needs of potential users. We conducted two focus groups and two participatory design workshops to explore the needs from healthcare professionals.

\section{Focus Groups}
We performed two focus groups to better understand practical needs of wearable technology in clinical. In first focus group, we recruited four nursing professionals and two physical therapists from National Taiwan University Hospital. In second focus group, we recruited four nursing professionals from Taipei City Psychiatric Center (TCPC).

\subsection{Participants}
In first focus group, we recruited four healthcare workers (four experienced nursing experts and two physiotherapists), aged from 27 to 46 years. The four experienced nursing experts had worked as registered nurses for more than three years and the two physiotherapists have been worked in the Rehabilitation Department of National Taiwan University Hospital at least six years. Those participants were experienced in divisions of oncology, cardiology, and orthopedics.
Four experts in second focus group, aged from 35 to 45 years, had worked as registered nurses for more than ten years at the Taipei City Psychiatric Center (TCPC) and specialized in providing inpatient psychiatric care for children, adolescents, geriatrics, and patients with substance abuse.

\subsection{Procedures}
The moderator of the focus group was a researcher with a background in engineering. After explaining the objective of the focus group, we provide five mock-up wearable devices with different form factors to encourage the participants to hands on. Then the moderator asked their perceptions regarding the usability, safety, and potential applications of those form factors. The second topic of the focus group is to encourage a story of the clinical scenario that can reveal any information derived from wearable device can benefit a healthcare application.

\section{Participatory Design Workshops}
We conducted the study to examine opportunities and constrains concerning the use of sensing technology, including wearable and environmental sensing technologies in healthcare applications. Following the participatory design process \cite{Greenbaum:1992:DWC:125470, Muller:2002:PDT:772072.772138}, two workshops were held to co-design devices to assist healthcare workers in designing a wearable sensing platform.

\subsection{Participants}
Eight nursing professionals (four experienced nursing experts and four senior nursing school students), aged from 22 to 46 years, with experience in clinical nursing, were recruited. The four experienced nursing experts had worked as registered nurses for at least seven years and the four senior nursing school students had been nursing interns in hospitals for at least half a year. All participants were experienced in nursing patients in multiple medical divisions, with the main specialty spanning in divisions of oncology, pediatrics, cardiology, and orthopedics.

\subsection{Prior to a Workshop}
The nursing professionals were briefly introduced to the goal of this study and the structure of the workshop several days (about four to six days) prior to the workshop. Before attending the workshop, the participants were asked to finish the pre-workshop homework just as “homework” was assigned to participants prior to each workshop in the PICTIVE project \cite{Muller:1991:PEP:108844.108896}. This pre-workshop homework involved a design sheet to help participants describe and/or sketch five potential ideas for designs of devices that used any related wearable sensing technology to assist them in resolving difficulties that they faced at work or in their daily life. Since the purpose of this pre-workshop homework was to encourage the participants to generate a wide range of designs in the workshop, the participants were not asked to evaluate the feasibility of their prototype devices for solving the described problems or to limit them in their focus on post-operative caring scenarios. This pre-workshop homework helped the participants to brainstorm a wide range of ideas and collect observations from their working or everyday life as a warm-up for the upcoming workshop.

\subsection{Procedures}
A workshop comprised three sessions, which were (1) an engagement session, (2) a guiding session, and (3) a development session. Two participatory design workshops were held on two separate days. Each workshop lasted approximately four hours, including a half-an-hour break for lunch. \vspace{10pt}

\textbf{Engagement session:} 
\newline
In the engagement session (which lasted for half of an hour), all participants introduced themselves and presented the five design ideas that they had prepared for their pre-workshop homework. While presenting a design idea, each participant stuck a design note that also described the idea on a white board. Researchers generated appropriate categories and grouped similar ideas as the participants were placing their notes. After all of the notes on the white board, researchers and participants discussed and moved the notes among categories or to a new category using the affinity diagramming method. All of the categories indicate potential directions or building blocks in the development of the solution in the final design of the workshop. 

\textbf{Guiding session:} 
\newline
In the guiding session (which lasted for half of an hour), a moderator, who was a researcher with an engineering background, introduced the sensing technologies or projects related to healthcare applications. The purpose of the workshop was explained to participants, which was to use wearable and/or environmental sensing technologies in the design of devices that can help healthcare workers in caring for post-operative patients in hospital. 

\textbf{Development session:} 
\newline
In the final developing session (which lasted two and a half hours), four participants were separated into two pairs in which each participant could share design opinions. To stir cross-disciplinary thinking within each pair, one researcher with an engineering background helped participants to identify alternative sensing technologies for use in their design work and another researcher with a background in industrial and commercial design expressed their design scenarios using concrete storyboards. Consistent with the work-oriented participatory design process, the goal of each design was set, which was to help health care workers care for patients in hospitals. To assist the participants in performing the design task effectively, an A3 sheet of paper with the following fields was provided; name of their solution, target user, needs of the target user, targeted sensing events and required feedback, form factor of the designed device. The first stage of the development session involved the participants discussing the name and target user of the designed solution using categories that were obtained from the pre-workshop homework. Following the identification of the target users, the second stage involved determining the potential needs of the target user that are associated with the solution. In the third stage, the engineering researcher discussed with participants the selection of sensors and actuators for detecting targeted events or generating the required feedback in each application scenario. When the group completed its final design, the designer cooperated with the participants to finalize the form factor of the designed device by sketching an appropriate storyboard. In the final stage, each group presented their final design, about which all participants discussed and commented. 

\section{Findings form Explorative Studies}
In order to collect every experience sharing, perspective and interaction in discussion for further analysis, the audio recordings of focus groups and the video recordings of the participatory design workshops were transcribed and coded for key findings by researchers. All finding of design exploration for developing a wearable sensing platform are summarized as follows.
\vspace{15pt}
\newline
\textbf{Form factor:}\newline
Several wearable form factors are discussed to identify what are the nice characteristics as a wearable device, including wristband, neckband, sticker, bandage, badge, garment, flip flop, strap and so on. The explored characteristics include easy wearing, taking off, compact and comfortable. The device should be comfort to wear and quick to deploy. The device also can be placed on proper body location to sense physiological data.
\vspace{10pt}
\newline
\textbf{Flexibility:}\newline
Because there are various types of physiological sensors that can be used to develop different healthcare applications, a fixed design can only to offer limited functions on a device. It is difficult to have one single design that can fit all requirements to develop healthcare application. Therefore, if a wearable sensing platform has flexibility that developers can build a wearable device with required sensing abilities. Then, developers can use this device as a prototype to test their tailored healthcare application. In different iterations of design process, a platform with flexibility can easily deal with the situation that developers may need to add or remove functions on the wearable device.
\vspace{10pt}
\newline
\textbf{Extensibility:}\newline
A wearable sensing platform must be easily extend and upgrade its abilities. In terms of the hardware part, the platform must be able extend new component without re-design entire system. In terms of the software part, the platform must provide comprehensive APIs to allow users to construct new method, function and algorithm on top of existing components.
\vspace{10pt}
\newline
\textbf{Sensing and data:}\newline
To develop a healthcare application, several types of common information are collected using wearable device, such as vital sign, physical activity and so on. Since a wearable device collects sensory data, the next step is to process the data. Then, the system can feedback the processed data to corresponding users depending on their requirement. The platform should provide a meaningful representation of data that can be understood by the user, e.g.: a pedometer shows step count rather than raw data of motion sensor.
\vspace{10pt}
\newline
\textbf{Configuration:}\newline
The complexity of configuration is a key whether a user will adopt a platform for development or not. In terms of non-engineering background users, they desire there is no complex steps to configure the device and the components are plug and play.
\vspace{10pt}
\newline
\textbf{Battery life:}\newline
The work period of the wearable device is a critical concern for a healthcare application. The device must be able to survive until its mission is complete. A proper power saving mechanism can prolong the work period of the wearable device as long as possible and can work transparently to the user without complex configuration.

