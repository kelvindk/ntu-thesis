%\begin{abstractzh}
%\end{abstractzh}

\begin{abstracten}
\noindent

The wearable devices for healthcare applications are a rapid growth market in recent years and are expected to further grow in the near future. The researchers also present many emerging healthcare projects with wearable device from a variety of disciplines. However, development of a wearable device needs much iteration to go though the steps between ideate, prototype and test. Different healthcare applications may need to sense diverse types of physiological data and/or activities and use different form factors. The iterative design process may take a lot of time, cost and effort to advance the prototype from low-fidelity to high-fidelity in order to achieve a right design. Prototyping also requires a lot of engineering skills to implement a device to validate the concept and hypothesis of the design. To address these challenges, this thesis proposes a wearable sensing platform that allows developers to rapidly prototype their health applications. We first consider the users of our wearable platform are researchers and designers who may not have sufficient engineering skills to build a prototype to test their design in developing phase. Toward an easy-to-use wearable platform for those users, we first conduct focus group and participatory design workshops to explore the needs from the users with medical background. We therefore design a wearable platform based on the key findings summarized from the design exploration studies, called the BioScope, which is a wearable sensing platform with flexibility and extensibility. Each sensor is one single patch in the platform. The flexibility enables users to choose the needed sensor patches and assemble them together like building LEGO bricks. The platform is extensible which can simplify the process to extend new component without re-design entire system. The platform provides several basic sensor patches that can easily collect physiological data from human body. An interaction patch has a compact display to quickly feedback sensory data and a gesture input sensor to simply control the wearable device. Bandage-like is the basic form factor to simply affix the device on arbitrary body location like a bandage. The platform has a library that contains the basic sensing and feedback functions of each patches that user can build their application that just simply select required sensing patches and corresponding feedback function. We also provide APIs on both wearable device and mobile device to serve advanced developer who attempts to construct customized function on existing patches.

\end{abstracten}

%\let\cleardoublepage\clearpage  




%\begin{comment}
%\category{I2.10}{Computing Methodologies}{Artificial Intelligence -- Vision and Scene Understanding} \category{H5.3}{InformationSystems}{Information Interfaces and Presentation (HCI) -- Web-based Interaction.}

%\terms{Design, Human factors, Performance.}

%\keywords{Region of interest, Visual attention model, Web-based games, Benchmarks.}
%\end{comment}
